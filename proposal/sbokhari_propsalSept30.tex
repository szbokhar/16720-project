\documentclass[11pt]{exam}
\usepackage{amsmath,amssymb,url,graphicx,tabularx,array,geometry,amsthm}

\begin{document}

\title{16-720 Project Proposal: Selected Object Area Tracking}
\date{}
\author{Syed Zahir Bokhari (AndrewID: sbokhari)}

\maketitle

\section{Abstract}
In many situations object tracking systems give only a position or bounding box
of the object being tracked. However there are many cases during tracking a
foreground object requires us to know all the pixels associated with the
forground objects. This is typically achived through a background subtraction
system, which will reveal all foreground pixels. However in outdoor and publics
spaces, the background is often dynamic, and contains much motion which we would
not consider foreground.

Allowing the user to specify a desired object (or objects) of interest by means
of a fuzzy selection will make it possible to discover the boundary of the
object, by menas of \cite{canny} or \cite{snakes}. Only one method should be
needed, however there is a tradeoff between accuracy and time.

Once the object is selected, \cite{bacsub} details a robust way to estimate the
background of outdoor scenes without foreknowledge of the background, and is
adaptive to long term changes in the scene. This should help in discovering the
changing boundery of the object as the video plays, and move the contour as
necessary.

\section{Technologies}
MATLAB for prototyping and proof of concept. Python and OpenCV for final
implementation if time permits.

\begin{thebibliography}{9}
\bibitem{canny}
J. Canny.
"A Computational Approach to Edge Detection,"
\emph{Pattern Analysis and Machine Intelligence, IEEE Transactions} on
(Volume:PAMI-8, Issue: 6)

\bibitem{snakes}
M. Kass, A. Witkin, D. Terzopoulos.
"Snakes: Active Contour Model,"
\emph{International Journal of Computer Vision}

\bibitem{bacsub}
C. Stauffer, W. Grimson.
"Adaptive background mixture models for real-time tracking,"
\emph{Computer Vision and Pattern Recognition, 1999. IEEE Computer Society
Conference} on. (Volume:2)

\end{thebibliography}

\end{document}
